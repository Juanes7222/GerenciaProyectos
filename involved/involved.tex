\section{Involucrados}

\begin{itemize}
    \item Usuarios
    \item Escuelas de conducción
    \item Centros de reconocimiento de conductores (CRC)
    \item Centros de enseñanza automovilística (CEA)
    \item RUNT (Registro Único Nacional de Tránsito)
    \item Ministerio de Transporte
    \item Secretaría de Tránsito de Pereira
    \item Entidades de salud (IPS)
    \item Agencias de seguros
    \item Proveedores de mensajería
    \item Gobernación de Risaralda y alcaldía local
    \item Entidades financieras
    \item Tramitadores
    \item AMCO (Área Metropolitana Centro Occidente)
    \item Asociación de Taxistas de Pereira
    \item ADITT (Asociación para el Desarrollo Integral del Transporte Terrestre Intermunicipal)
\end{itemize}
 

\section{Grupos de Involucrados}

\begin{itemize}
    \item \textbf{Particulares:}  
    Incluyen a las personas y actores individuales que interactúan directamente con el sistema de tránsito.  
        \begin{itemize}
            \item \textbf{Usuarios:} Personas que requieren tramitar su licencia. Son la población directamente beneficiada y quienes hacen uso de los servicios para poder conducir de forma legal.  
            \item \textbf{Tramitadores:} Personas que intermedian en procesos de tránsito para facilitar o agilizar la gestión documental de terceros.  
        \end{itemize}

    \item \textbf{Formación y certificación:}  
    Son las instituciones que garantizan la preparación y validación de las competencias necesarias para conducir.  
        \begin{itemize}
            \item \textbf{Escuelas de conducción:} Instituciones autorizadas para formar conductores en conocimientos teóricos y prácticos, asegurando que adquieran las competencias necesarias para manejar de manera segura.  
            \item \textbf{Centros de reconocimiento de conductores (CRC):} Entidades que realizan evaluaciones médicas, psicológicas y físicas para certificar que un aspirante está en condiciones de conducir.  
            \item \textbf{Centros de enseñanza automovilística (CEA):} Organizaciones que refuerzan la formación vial mediante cursos de actualización, reentrenamiento y programas complementarios.  
            \item \textbf{Entidades de salud (IPS):} Centros médicos habilitados para evaluar condiciones físicas, visuales y psicológicas de los aspirantes.  
        \end{itemize}

    \item \textbf{Estado:}  
    Entidades públicas responsables de regular, supervisar y garantizar el cumplimiento de las normas de tránsito.  
        \begin{itemize}
            \item \textbf{Ministerio de Transporte:} Entidad encargada de formular políticas, reglamentar y supervisar el sistema de transporte a nivel nacional.  
            \item \textbf{RUNT:} Base de datos nacional que centraliza información sobre conductores, vehículos y trámites, garantizando control y trazabilidad en el sistema de tránsito.  
            \item \textbf{Secretaría de Tránsito de Pereira:} Autoridad local responsable de trámites, sanciones, expedición de documentos y control de movilidad en la ciudad.  
            \item \textbf{Gobernación y alcaldía local:} Entidades territoriales que coordinan políticas de movilidad y tránsito en el departamento y municipio.  
        \end{itemize}

    \item \textbf{Empresas:}  
    Actores del sector privado que prestan servicios complementarios para el funcionamiento del sistema de tránsito.  
        \begin{itemize}
            \item \textbf{Agencias de seguros:} Compañías que ofrecen pólizas obligatorias y voluntarias relacionadas con el tránsito, como el SOAT.  
            \item \textbf{Entidades financieras:} Bancos y pasarelas de pago que permiten realizar las transacciones asociadas a trámites de tránsito.  
            \item \textbf{Proveedores de mensajería:} Empresas que apoyan en la entrega de documentos físicos como licencias y certificaciones.  
        \end{itemize}

    \item \textbf{Gremios:}  
    Asociaciones y colectivos que representan intereses de grupos de transporte y movilidad.  
        \begin{itemize}
            \item \textbf{AMCO:} Entidad pública que coordina proyectos de movilidad y transporte en Pereira, Dosquebradas y La Virginia.  
            \item \textbf{Asociación de Taxistas de Pereira:} Gremio que agrupa y representa a los conductores de taxi de la ciudad.  
            \item \textbf{ADITT:} Organización que representa a empresas y conductores del transporte intermunicipal.  
        \end{itemize}
\end{itemize}
  

\section{Mapa de involucrados (Mentefacto)}
\begin{figure}[H]
    \centering
    \includegraphics[width=1\linewidth]{involved/images/Gerencia_1.png}
    \caption{Mapa de Involucrados (Mentefacto)}
    \label{fig:placeholder}
    \vspace{0.2cm}
    \small Fuente: Elaboración propia.
\end{figure}

\section{Matriz de Problemas - Intereses Percibidos}
\begin{table}[H]
\centering
\begin{tabular}{|p{4cm}|p{5cm}|p{6cm}|}
\hline
\textbf{Involucrado} & \textbf{Interés} & \textbf{Problema} \\ \hline
Usuarios & Acceder a trámites de licencias de forma rápida y clara. & Trámites lentos y confusos. \\ \hline
Escuelas de conducción & Aumentar visibilidad y captar más estudiantes. & Baja difusión y exceso de competencia informal. \\ \hline
CRC & Más usuarios agendando exámenes médicos. & Desorden en citas y pérdida de usuarios. \\ \hline
CEA & Agilizar inscripción de aspirantes. & Procesos manuales que generan demoras. \\ \hline
RUNT & Garantizar trazabilidad y confiabilidad de la información. & Riesgo de falsificación y errores en registros. \\ \hline
Ministerio de Transporte & Mejorar digitalización y control. & Burocracia y lentitud en adopción digital. \\ \hline
Secretarías de Tránsito & Descongestionar oficinas. & Alta demanda presencial e ineficiencia. \\ \hline
IPS certificadoras & Optimizar atención de usuarios. & Retrasos y saturación de pacientes. \\ \hline
Agencias de seguros & Ofrecer pólizas asociadas a licencias. & Baja integración tecnológica. \\ \hline
Mensajería & Incrementar servicios de entrega. & Usuarios deben desplazarse a oficinas. \\ \hline
Gobernación y alcaldías & Modernización tecnológica. & Quejas por trámites presenciales lentos. \\ \hline
Entidades financieras & Facilitar pagos digitales. & Pagos presenciales con retrasos. \\ \hline
Tramitadores & Mantener ingresos. & Pérdida de demanda de sus servicios. \\ \hline
AMCO & Implementar procesos ágiles. & Persistencia de procesos lentos. \\ \hline
Asociación de Taxistas & Trámites ágiles y simples. & Trámites demorados y engorrosos. \\ \hline
ADITT & Trámites eficientes y accesibles. & Procesos ineficientes y desactualizados. \\ \hline
\end{tabular}
\caption{Matriz de problemas}

\end{table}


\section{Matriz de Expectativas - Fuerza}
\begin{table}[h]
\centering
\begin{tabular}{|p{4cm}|c|c|c|c|}
\hline
\textbf{Involucrado} & \textbf{Expectativa} & \textbf{Fuerza} & \textbf{Resultado} & \textbf{Rol} \\ \hline
Usuarios & 5 & 3 & 15 & Favorecedor \\ \hline
Escuelas de conducción & 3 & 3 & 9 & Favorecedor \\ \hline
CRC & 2 & 3 & 6 & Neutral \\ \hline
CEA & 2 & 3 & 6 & Neutral \\ \hline
RUNT & 4 & 5 & 20 & Favorecedor \\ \hline
Ministerio de Transporte & 4 & 5 & 20 & Favorecedor \\ \hline
Secretarías de Tránsito & 3 & 5 & 15 & Favorecedor \\ \hline
IPS certificadoras & 2 & 3 & 6 & Neutral \\ \hline
Agencias de seguros & 1 & 2 & 2 & Neutral \\ \hline
Mensajería & 1 & 2 & 2 & Neutral \\ \hline
Gobernación y alcaldías & 3 & 4 & 12 & Favorecedor \\ \hline
Entidades financieras & 2 & 3 & 6 & Neutral \\ \hline
Tramitadores informales & -5 & 1 & -5 & Neutral \\ \hline
AMCO & 4 & 3 & 12 & Favorecedor \\ \hline
Asociación de Taxistas & 4 & 3 & 12 & Favorecedor \\ \hline
ADITT & 4 & 3 & 12 & Favorecedor \\ \hline
\end{tabular}
\end{table}

\section{Análisis de Involucrados}
\justify
En primer lugar, los usuarios constituyen el grupo central, pues son los directamente
beneficiados con el acceso a trámites más rápidos, claros y económicos. Con una alta
expectativa y una fuerza considerable, su papel es claramente favorecedor, dado que
demandan soluciones que reduzcan costos y tiempos. Con la propuesta de licencia virtual
con código QR y validación biométrica, los usuarios se convierten en los principales
promotores del sistema, ya que podrán acceder a su documento de forma inmediata y
segura, sin depender de intermediarios.

Las escuelas de conducción, secretarías de tránsito, el RUNT, el Ministerio de
Transporte y las autoridades locales se ubican también como actores favorecedores. Estos
buscan mejorar la visibilidad de sus servicios, garantizar trazabilidad de la información,
digitalizar procesos y descongestionar las oficinas. La incorporación de un sistema
centralizado permitirá que el RUNT valide automáticamente la autenticidad de los datos,
que los CRC integren en línea los resultados de exámenes médicos y que las academias de
conducción registren el avance de los estudiantes, lo que fortalece la transparencia y control
institucional.

Por otra parte, entidades como los CRC, CEA, IPS certificadoras, agencias de
seguros, proveedores de mensajería y entidades financieras presentan un rol más neutral. Si
bien tienen intereses particulares en agilizar procesos o ampliar sus servicios, su influencia
es limitada y dependen de las decisiones de los actores principales para mejorar su
integración. En el nuevo esquema digital, estos actores deberán adaptarse a plataformas
automatizadas, lo que puede representar un reto tecnológico pero también una oportunidad
de modernización.

En cuanto a los gremios del transporte como AMCO, la Asociación de Taxistas de Pereira y la ADITT, muestran un papel favorecedor al representar colectivos que buscan
mayor organización, seguridad y eficiencia en la movilidad. Con la licencia digital, estos
gremios se benefician de un control más estricto de la legalidad de los conductores y la
reducción de fraudes en la expedición de documentos.

Finalmente, se identifican los tramitadores informales como un grupo detractor.
Estos actores ven amenazada su labor debido a la digitalización y simplificación de
trámites, pues el sistema reduce la necesidad de intermediación. Aunque su fuerza es
limitada, representan un riesgo potencial en la medida en que pueden generar resistencia o
prácticas indebidas.

En conclusión, la mayoría de los grupos de interés se alinean como favorecedores, lo
cual indica una alta probabilidad de éxito en la implementación del aplicativo tramite de licencia
virtual, siempre que se logre articular la participación de los diferentes niveles de gobierno,
instituciones educativas, gremios y usuarios finales. Los actores neutrales pueden
convertirse en aliados estratégicos si se integran adecuadamente, mientras que los
detractores deberán gestionarse con estrategias que mitiguen su resistencia. El uso de
tecnologías como el QR dinámico, la biometría y la centralización en el RUNT fortalece la
confianza del sistema y marca una transición hacia la modernización total de los trámites
de tránsito en Pereira.
