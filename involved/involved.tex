\section{Involucrados}

\begin{itemize}
	\item Usuarios
	\item Escuelas de conducción
	\item Centros de reconocimiento de conductores (CRC)
	\item Centros de enseñanza automovilística (CEA)
	\item RUNT (Registro Único Nacional de Tránsito)
	\item Ministerio de Transporte
	\item Secretaría de Tránsito de Pereira
	\item Entidades de salud (IPS)
	\item Agencias de seguros
	\item Proveedores de mensajería
	\item Gobernación de Risaralda y alcaldía local
	\item Entidades financieras
	\item Tramitadores
	\item AMCO (Área Metropolitana Centro Occidente)
	\item Asociación de Taxistas de Pereira
	\item ADITT (Asociación para el Desarrollo Integral del Transporte Terrestre Intermunicipal)
\end{itemize}


\section{Grupos de Involucrados}

\begin{itemize}
	\item \textbf{Particulares:}
	      Incluyen a las personas y actores individuales que interactúan directamente con el sistema de tránsito.
	      \begin{itemize}
		      \item \textbf{Usuarios:} Personas que requieren tramitar su licencia. Son la población directamente beneficiada y quienes hacen uso de los servicios para poder conducir de forma legal.
		      \item \textbf{Tramitadores:} Personas que intermedian en procesos de tránsito para facilitar o agilizar la gestión documental de terceros.
	      \end{itemize}

	\item \textbf{Formación y certificación:}
	      Son las instituciones que garantizan la preparación y validación de las competencias necesarias para conducir.
	      \begin{itemize}
		      \item \textbf{Escuelas de conducción:} Centros formativos que imparten enseñanza teórica y práctica para obtener permisos o licencias de conducción. Suelen contar con circuitos de práctica, vehículos y profesorado especializado (Wikipedia, s.f.).
		      \item \textbf{Centros de reconocimiento de conductores (CRC):} Entidades que realizan evaluaciones médicas, psicológicas y físicas para certificar que un aspirante está en condiciones de conducir (Centro de Reconocimiento de Conductores CRC,2023).
		      \item \textbf{Centros de enseñanza automovilística (CEA):} Organizaciones que refuerzan la formación vial mediante cursos de actualización, reentrenamiento y programas complementarios (Centro de Enseñanza Automovilística (CEA), s/f).
		      \item \textbf{Entidades de salud (IPS):} Establecimientos (públicos, privados, mixtos o comunitarios) autorizados para prestar servicios clínicos, hospitalarios y de atención médica directamente a los usuarios del POS (Ley 100 de 1993, art. 156; Corte Constitucional, s. f.; Bogotá.gov.co, 2021).
	      \end{itemize}

	\item \textbf{Estado:}
	      Entidades públicas responsables de regular, supervisar y garantizar el cumplimiento de las normas de tránsito (Colombia, s/f).
	      \begin{itemize}
		      \item \textbf{Ministerio de Transporte:} Entidad encargada de formular políticas, reglamentar y supervisar el sistema de transporte a nivel nacional (Mintransporte, 2011).
		      \item \textbf{RUNT:} Base de datos nacional que centraliza información sobre conductores, vehículos y trámites, garantizando control y trazabilidad en el sistema de tránsito (¿Qué es el RUNT?, s/f).
		      \item \textbf{Secretaría de Tránsito de Pereira:} Secretaría de Tránsito de Pereira: Entidad municipal responsable de coordinar el cumplimiento del plan de tránsito, tramitar procedimientos técnicos y gestionar cultura vial en Pereira (Pereira, 2023).
		      \item \textbf{Gobernación y alcaldía local:} La Gobernación de Risaralda es la entidad del gobierno departamental encargada de administrar y coordinar los asuntos públicos del departamento. La Alcaldía de Pereira es la entidad municipal que representa el gobierno local de la ciudad (Federación Nacional de Departamentos, s. f.).
	      \end{itemize}

	\item \textbf{Empresas:}
	      Actores del sector privado que prestan servicios complementarios para el funcionamiento del sistema de tránsito.
	      \begin{itemize}
		      \item \textbf{Agencias de seguros:} Empresas que actúan como intermediarias entre los clientes (personas o empresas) y las compañías aseguradoras. Asesoran, venden y gestionan pólizas, captan nuevos clientes y renuevan pólizas existentes (Sigo Seguros, 2024).
		      \item \textbf{Entidades financieras:} Instituciones (como bancos) que gestionan servicios financieros: administración y préstamo de dinero, manejo de cuentas, inversión y crédito (Fondo de Pensiones de Colombia [FOPEP], s. f.).
		      \item \textbf{Proveedores de mensajería:} Empresas especializadas en recoger, transportar y entregar documentos o paquetes. El servicio expreso (o Courier) opera con rapidez y cubre envíos puerta a puerta, a nivel nacional o internacional, con seguimiento y eficiencia (Inter Rapidísimo, s. f.)
		      \item \textbf{AMCO:} Entidad pública que coordina proyectos de movilidad y transporte en Pereira, Dosquebradas y La Virginia (Occidente, 2020).
		      \item \textbf{Asociación de Taxistas de Pereira:} Entidad sin ánimo de lucro conformada por conductores de taxi para defender sus derechos y promover proyectos comunes (Informacolombia, s. f.).
		      \item \textbf{ADITT:} Organización que representa a empresas y conductores del transporte intermunicipal (Conózcanos, s/f).
	      \end{itemize}
\end{itemize}


\section{Mapa de involucrados (Mentefacto)}
\begin{figure}[H]
	\centering
	\includegraphics[width=1\linewidth]{involved/images/Gerencia_1.png}
	\caption{Mapa de Involucrados (Mentefacto)}
	\label{fig:placeholder}
	\vspace{0.2cm}
	\small Fuente: Elaboración propia.
\end{figure}

\section{Matriz de Problemas - Intereses Percibidos}
\begin{table}[H]
	\centering
	\begin{tabular}{|p{4cm}|p{5cm}|p{6cm}|}
		\hline
		\textbf{Involucrado}     & \textbf{Interés}                                           & \textbf{Problema}                               \\ \hline
		Usuarios                 & Trámites rápidos y claros                                  & Trámites lentos y confusos.                     \\ \hline
		Escuelas de conducción   & Aumentar visibilidad y captar más estudiantes.             & Escasa visibilidad y pérdida de estudiantes. \\ \hline
		CRC                      & Incrementar agendamiento de exámenes médicos.              & Ineficiencia en agendamiento de citas médicas.        \\ \hline
		CEA                      & Agilizar inscripción de aspirantes.                        & Inscripción lenta y engorrosa de aspirantes.          \\ \hline
		RUNT                     & Registros no duplicados y alta trazabilidad.               & Registros duplicados y baja trazabilidad. \\ \hline
		Ministerio de Transporte & Mejorar digitalización y control.                          & Digitalización deficiente y no bajo control.      \\ \hline
		Secretarías de Tránsito  & Descongestionar oficinas.                                  & Congestión presencial por gestión ineficiente.         \\ \hline
		IPS certificadoras       & Optimizar atención de usuarios.                            & Saturación y demoras en atención a los usuarios.           \\ \hline
		Agencias de seguros      & Ofrecer pólizas asociadas a licencias.                     & Inexistencia de integración de oferta de pólizas para licencias.                   \\ \hline
		Mensajería               & Alta demanda de mensajería.                          & Baja demanda de mensajería por trámites presenciales.      \\ \hline
		Gobernación y alcaldías  & Lograr modernización tecnológica.                                & Rezago tecnológico en la gestión pública \\ \hline
		Entidades financieras    & Facilitar pagos digitales.                                 & Pagos digitales limitados y trámites presenciales.                \\ \hline
		Tramitadores             & Mantener ingresos por gestión de trámites.                                         & Reducción de ingresos por digitalización parcial.            \\ \hline
		AMCO                     & Implementar procesos ágiles.                               & Resistencia al cambio y baja capacitación.                \\ \hline
		Asociación de Taxistas   & Trámites ágiles y simples para conductores.                 & Trámites demorados y engorrosos.                \\ \hline
		ADITT                    & Trámites eficientes y de fácil acceso.                       & Trámites ineficientes y de difícil acceso.        \\ \hline
	\end{tabular}
	\caption{Matriz de problemas}
    \vspace{0.3cm}
    \small Fuente: Elaboración propia.

\end{table}


\section{Matriz de Expectativas - Fuerza}
\begin{table}[h]
	\centering
	\begin{tabular}{|p{4cm}|c|c|c|c|}
		\hline
		\textbf{Involucrado}     & \textbf{Expectativa} & \textbf{Fuerza} & \textbf{Resultado} & \textbf{Rol} \\ \hline
		Usuarios                 & 5                    & 3               & 15                 & Favorecedor  \\ \hline
		Escuelas de conducción   & 3                    & 3               & 9                  & Favorecedor  \\ \hline
		CRC                      & 2                    & 3               & 6                  & Neutral      \\ \hline
		CEA                      & 2                    & 3               & 6                  & Neutral      \\ \hline
		RUNT                     & 4                    & 5               & 20                 & Favorecedor  \\ \hline
		Ministerio de Transporte & 4                    & 5               & 20                 & Favorecedor  \\ \hline
		Secretarías de Tránsito  & 3                    & 5               & 15                 & Favorecedor  \\ \hline
		IPS certificadoras       & 2                    & 3               & 6                  & Neutral      \\ \hline
		Agencias de seguros      & 1                    & 2               & 2                  & Neutral      \\ \hline
		Mensajería               & 1                    & 2               & 2                  & Neutral      \\ \hline
		Gobernación y alcaldías  & 3                    & 4               & 12                 & Favorecedor  \\ \hline
		Entidades financieras    & 2                    & 3               & 6                  & Neutral      \\ \hline
		Tramitadores informales  & -5                   & 1               & -5                 & Neutral      \\ \hline
		AMCO                     & 4                    & 3               & 12                 & Favorecedor  \\ \hline
		Asociación de Taxistas   & 4                    & 3               & 12                 & Favorecedor  \\ \hline
		ADITT                    & 4                    & 3               & 12                 & Favorecedor  \\ \hline
	\end{tabular}
	\caption{Matriz de expectativas - fuerza}
    \vspace{0.3cm}
    \small Fuente: Elaboración propia.
\end{table}

\section{Análisis de Involucrados}
\justify
Los usuarios tienen el foco de su interés puesto en el desarrollo de trámites rápidos y claros, dado que en la actualidad los trámites son lentos y confusos. Éste grupo presenta una muy alta expectativa, calificado en la matriz de expectativas y fuerza con un puntaje de 5. En términos de la fuerza, dicho grupo tiene una calificación intermedia, puntuado con un valor de 3, lo cual determina un rol de favorecedor, para el proyecto. 
Las escuelas de conducción presentan su interés en el incremento de la visibilidad y la captación de una mayor cantidad de estudiantes, dado que en la actualidad existe una escasa visibilidad y alta pérdida de alumnos. Éste grupo presenta una expectativa intermedia, calificado en la matriz de expectativas y fuerza con un puntaje de 3. En términos de la fuerza, dicho grupo tiene una calificación intermedia, puntuado con un valor de 3, lo cual determina un rol de favorecedor, para el proyecto. 
Los CRC tienen el foco de su interés puesto en la agilización de inscripción de aspirantes, dado que en la actualidad se presenta una clara ineficiencia en dicho proceso de agendamiento. Éstas entidades presentan una baja expectativa, calificado en la matriz de expectativas y fuerza con un puntaje de 2. En términos de la fuerza, dicho grupo tiene una calificación intermedia, puntuado con un valor de 3, lo cual determina un rol de neutral, para el proyecto. 
Los CEA tienen el foco de su interés puesto en el aumento del agendamiento de exámenes médicos, puesto que se presenta un proceso de inscripción lento y engorroso de los aspirantes. Éstos centros presentan una baja expectativa, calificado en la matriz de expectativas y fuerza con un puntaje de 2. En términos de la fuerza, dicho grupo tiene una calificación intermedia, puntuado con un valor de 3, lo cual determina un rol de neutral, para el proyecto. 
El RUNT presenta su foco de interés en la gestión de registros no duplicados y dar garantías de alta trazabilidad, teniendo en cuenta que en la actualidad, los registros presentan una elevada duplicación y la trazabilidad es baja. Éste sistema presenta una alta expectativa, calificado en la matriz de expectativas y fuerza con un puntaje de 4. En términos de la fuerza, dicho grupo tiene una calificación alta, puntuado con un valor de 5, lo cual determina un rol de favorecedor, para el proyecto. 
El Ministerio de Transporte tiene su interés puesto en la mejora de la digitalización y el control, teniendo en cuenta que en la actualidad, se desarrolla una digitalización deficiente y un bajo control. Ésta entidad presenta una alta expectativa, calificado en la matriz de expectativas y fuerza con un puntaje de 4. En términos de la fuerza, dicho grupo tiene una calificación alta, puntuado con un valor de 5, lo cual determina un rol de favorecedor, para el proyecto. 
Las secretarías de tránsito tienen su interés puesto en la descongestión de las oficinas, dado que en la actualidad, se presenta una alta congestión presencial, por una gestión ineficiente. Éstas entidades presentan una expectativa intermedia, calificado en la matriz de expectativas y fuerza con un puntaje de 3. En términos de la fuerza, dicho grupo tiene una calificación alta, puntuado con un valor de 5, lo cual determina un rol de favorecedor, para el proyecto. 
Las IPS certificadoras de tránsito tienen especial énfasis en la optimización de la atención de los usuarios, teniendo en cuenta la elevada saturación y demoras en la atención. Éstas entidades presentan una expectativa baja, calificado en la matriz de expectativas y fuerza con un puntaje de 2. En términos de la fuerza, dicho grupo tiene una calificación intermedia, puntuado con un valor de 3, lo cual determina un rol de neutral, para el proyecto. 
Las agencias de seguros tienen interés en el ofrecimiento de pólizas asociadas a las licencias, dado a la nula integración de oferta de pólizas para licencias. Éstas entidades presentan una expectativa baja, calificado en la matriz de expectativas y fuerza con un puntaje de 1. En términos de la fuerza, dicho grupo tiene una calificación baja, puntuado con un valor de 2, lo cual determina un rol de neutral, para el proyecto. 
La mensajería presenta un interés en la alta demanda de mensajes, teniendo en cuenta la baja demanda de mensajería en trámites presenciales. Ésta entidad presenta una expectativa baja, calificado en la matriz de expectativas y fuerza con un puntaje de 1. En términos de la fuerza, dicho grupo tiene una calificación baja, puntuado con un valor de 2, lo cual determina un rol de neutral, para el proyecto.  
La gobernación y alcaldías tienen su interés puesto en lograr una modernización tecnológica, debido al rezago tecnológico en el que se ha visto la gestión pública. Éstos entes presentan una expectativa intermedia, calificado en la matriz de expectativas y fuerza con un puntaje de 3. En términos de la fuerza, dicho grupo tiene una calificación alta, puntuado con un valor de 4, lo cual determina un rol de favorecedor, para el proyecto.  
Las entidades financieras presentan su enfoque en proveer pagos digitales, puesto que en la actualidad existe una limitación en cuanto a pagos digitales y una tediosa necesidad de trámites presenciales. Dichas entidades presentan una expectativa baja, calificado en la matriz de expectativas y fuerza con un puntaje de 2. En términos de la fuerza, dicho grupo tiene una calificación intermedia, puntuado con un valor de 3, lo cual determina un rol de neutral, para el proyecto. 
Los tramitadores informales tienen interés en conservar sus ingresos por la gestión de los trámites, teniendo en cuenta que una digitalización parcial, representaría una disminución de sus ingresos. Éste grupo presenta una expectativa muy baja, calificado en la matriz de expectativas y fuerza con un puntaje de -5. En términos de la fuerza, dicho grupo tiene una calificación baja, puntuado con un valor de 1, lo cual determina un rol de neutral, para el proyecto. 
El AMCO presenta un claro interés en la implementación de procesos ágiles, en vista de la evidente resistencia al cambio y la baja capacitación. La entidad presenta una alta expectativa, calificado en la matriz de expectativas y fuerza con un puntaje de 4. En términos de la fuerza, dicho grupo tiene una calificación intermedia, puntuado con un valor de 3, lo cual determina un rol de favorecedor, para el proyecto. 
La asociación de taxistas tiene un gran interés en la ejecución de trámites ágiles y simples para los conductores, a causa de la existencia de requisitos redundantes y trámites sumamente engorrosos para los conductores. La entidad presenta una alta expectativa, calificado en la matriz de expectativas y fuerza con un puntaje de 4. En términos de la fuerza, dicho grupo tiene una calificación intermedia, puntuado con un valor de 3, lo cual determina un rol de favorecedor, para el proyecto. 

La ADITT tiene un gran interés en el desarrollo de trámites eficientes y  de fácil acceso, por causa de la ejecución de trámites ineficientes y de difícil acceso. La asociación presenta una alta expectativa, calificado en la matriz de expectativas y fuerza con un puntaje de 4. En términos de la fuerza, dicho grupo tiene una calificación intermedia, puntuado con un valor de 3, lo cual determina un rol de favorecedor, para el proyecto. 

