\section{Arbol de Problemas}

\begin{figure}[H]
    \centering
    \includegraphics[width=0.9\linewidth]{trees/images/arbol_problemas.png}
    \caption{Arbol de Problemas.}
    \label{fig:placeholder}
    \vspace{0.2cm}
    \small Fuente: Elaboración propia.
\end{figure}

\subsection{Fundamentos del Árbol de Problemas}
\begin{itemize}

    \item \textbf{Poca integración locativa} \\
    En Pereira, la falta de articulación entre las oficinas de tránsito y la infraestructura urbana ha generado congestión y demoras en la atención. El crecimiento del parque automotor (43,4\% en 7 años) ha sobrepasado la capacidad física de los puntos de atención, lo que contribuye al caos vial y a la saturación de servicios.  
    \textbf{(El Pereirano, 2023)}

    \item \textbf{Poca productividad de los funcionarios} \\
    Los trámites para obtener licencias de conducción en Pereira pueden tardar entre uno y dos meses, incluso para procesos simples como la recategorización. Aunque existen plataformas como el RUNT, el proceso sigue siendo largo y requiere múltiples pasos presenciales, lo que refleja una burocracia persistente.  
    \textbf{(PaseDeMoto, 2025)}

    \item \textbf{Falta de personal de apoyo} \\
    El Instituto de Movilidad de Pereira enfrenta un déficit de al menos 120 agentes de tránsito, lo que limita la capacidad operativa para atender trámites y controlar el tráfico. Esta escasez de personal afecta directamente la eficiencia de los servicios y la seguridad vial.  
    \textbf{(RCN Radio, 2023; El Pereirano, 2023)}

    \item \textbf{Poca infraestructura tecnológica} \\
    Aunque el Instituto de Movilidad de Pereira ha invertido más de \$3.650 millones en modernización, los sistemas siguen mostrando rezagos que dificultan la eficiencia de los trámites.  
    \textbf{(El Pereirano, 2023)}

    \item \textbf{Procedimientos manuales} \\
    A pesar de los avances digitales, muchos trámites para licencias de tránsito en Risaralda aún dependen de procesos manuales, como entrega física de documentos, revisión operativa y archivo en libros. Esto ralentiza el flujo de atención y aumenta el margen de error.  
    \textbf{(PaseDeMoto, 2025)}

    \item \textbf{Problemática principal: Trámites para licenciaslentos y tediosos} \\
    En Pereira y el Eje Cafetero, la gestión de licencias de conducción presenta demoras constantes y procesos engorrosos. Se han registrado largas filas desde la madrugada en el Instituto de Movilidad para renovar el pase, reflejando la limitada capacidad operativa. A esto se suman las frecuentes caídas del RUNT, que en 2024 paralizaron trámites como expedición de licencias, renovaciones y cursos pedagógicos.  
    \textbf{(Concejo de Pereira, 2023; RCN Radio, 2024)}

    \item \textbf{Aumento de los reprocesos para los usuarios} \\
    Las fallas en el RUNT y la falta de interoperabilidad entre entidades generan reprocesos continuos. Cuando la plataforma entra en mantenimiento o falla, los usuarios deben reagendar servicios, volver a presentar documentos o repetir exámenes médicos, lo cual aumenta los costos emocionales y económicos.  
    \textbf{(Infobae, 2024)}

    \item \textbf{Pérdida productiva para los ciudadanos} \\
    El tiempo invertido en gestiones burocráticas reduce la productividad laboral. En promedio, una empresa en Colombia destina 2.620 horas al año a trámites administrativos, equivalente a un empleado de tiempo completo. En regiones como Risaralda, con abundantes trabajadores independientes y microempresarios, esta carga limita el tiempo dedicado a actividades productivas.  
    \textbf{(La República, 2025)}

    \item \textbf{Baja del PIB local} \\
    La acumulación de problemas burocráticos impacta negativamente el crecimiento económico. Colombia ha perdido posiciones en índices de competitividad internacional. En Risaralda, donde el transporte es clave para la economía, estos obstáculos reducen el impulso productivo.  
    \textbf{(La República, 2025)}

    \item \textbf{Aumento de costos para el Estado} \\
    La ineficiencia burocrática incrementa el gasto público. La necesidad de contratar personal eventual o mantener sistemas anticuados genera costos adicionales no previstos, reduciendo recursos disponibles para inversión en infraestructura o modernización tecnológica.  
    \textbf{(La República, 2025)}

    \item \textbf{Disminución del desempeño de inversión para el desarrollo} \\
    Los retrasos en la emisión de licencias afectan proyectos de infraestructura y movilidad. Grandes obras han quedado entrampadas en trámites administrativos, lo que conlleva sobrecostos, retrasos y menor confianza en las instituciones.  
    \textbf{(Valora Analitik, 2025)}
\end{itemize}

\section{Arbol de Soluciones}

\begin{figure}[H]
    \centering
    \includegraphics[width=1\linewidth]{trees/images/Arbol_Soluciones.png}
    \caption{Arbol de Soluciones.}
    \label{fig:placeholder}
     \vspace{0.2cm}
    \small Fuente: Elaboración propia.
\end{figure}
