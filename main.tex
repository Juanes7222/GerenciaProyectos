\documentclass[12pt, letterpaper]{article}

% =====================
% Paquetes
% =====================
\usepackage[utf8]{inputenc}
\usepackage[T1]{fontenc}
\usepackage[spanish]{babel}
\usepackage{lmodern}
\usepackage{setspace}
\usepackage[hyphens]{url}
\usepackage{amsmath}
\usepackage{geometry}
\usepackage{longtable}
\usepackage{booktabs}
\usepackage{fancyhdr}
\usepackage{array}
\usepackage{titlesec}
\usepackage{graphicx}
\usepackage{caption}
\usepackage{float}
\usepackage[style=apa, backend=biber]{biblatex}
\addbibresource{referencias.bib} % Archivo de bibliografía
\usepackage{xcolor}
\usepackage{hyphenat}
\usepackage{ragged2e} 
\usepackage{placeins}
\usepackage{csquotes}
\geometry{margin=1in}
\setlength{\parindent}{1.27cm}
\setlength{\parskip}{1em}
\onehalfspacing

% Configuración títulos
\titleformat{\section}{\normalfont\bfseries\uppercase}{\thesection.}{1em}{}
\titleformat{\subsection}{\normalfont\bfseries}{\thesubsection.}{1em}{}
\renewcommand{\contentsname}{Índice}

% Encabezados y pies
\pagestyle{fancy}
\fancyhf{}
\fancyfoot[C]{\thepage}

% =====================
% Documento
% =====================
% load hyperref as the last package to avoid redefinition/hook issues
\usepackage[hidelinks]{hyperref}
\let\cleardoublepage\clearpage
\begin{document}

% Portada
\begin{titlepage}
	\centering
	\includegraphics[width=5cm]{logo_universidad.png}\par
	\vspace{1cm}

	{\bfseries\LARGE Tercera entrega: Análisis de Alternativas\par}
	\vspace{2cm}
	{\large Autores: Jorge Luis Araque, Juan Camilo Velasquez, Juan David Tabares, Jhoan Esteban Corrales, Juan Esteban Cardona, Maicol Stiven Ruiz y Santiago Marin\par}
	{\large Docente: Alexánder Quintero\par}
	\vfill
	{\Large Facultad de Ingenierías, Universidad Tecnológica de Pereira \par}
	{\large Curso: Gerencia de Proyectos\par}
	{\large Pereira, Colombia \\ \today \par}
\end{titlepage}

\renewcommand{\contentsname}{Tabla de contenido}
% Índice
\tableofcontents
\newpage

\section{Matriz de Componentes, actividades, responsables y costos.}

\renewcommand{\arraystretch}{1.2} % Espaciado entre filas
\setlength{\tabcolsep}{5pt}       % Espaciado entre columnas

\begin{longtable}{|p{3.5cm}|p{6.5cm}|p{4.5cm}|}
	\caption{Matriz de Componentes, actividades, responsables y costos.}\label{tab:matriz}               \\
	\hline
	\textbf{Componente}                                      & \textbf{Actividad} & \textbf{Responsable} \\ \hline
	\endfirsthead
	\hline
	\textbf{Componente}                                      & \textbf{Actividad} & \textbf{Responsable} \\ \hline
	\endhead
	\hline
	\multicolumn{3}{r}{\textit{Continúa en la siguiente página}}                                         \\ \hline
	\endfoot
	\hline
	\multicolumn{3}{c}{\textit{Fuente: Elaboración propia.}}                                             \\ \hline
	\endlastfoot

	Procedimientos \newline automáticos                      &
	- Identificación de procesos \newline
	- Diseño de flujos automatizados \newline
	- Desarrollo de software  \newline
	- Pruebas y capacitación                                 &
	Ing. Sistemas \newline Grupo de Desarrollo \newline Jefes de Área                                    \\ \hline

	Alta integración  \newline locativa                      &
	- Diagnóstico de espacios físicos \newline
	- Reestructuración de áreas de \newline trabajo \newline
	- Implementación de mobiliario \newline ergonómico y tecnológico \newline
	- Adecuaciones eléctricas y de red                       &
	Área de Infraestructura \newline Arquitecto \newline Proveedores externos                            \\ \hline

	Alta productividad de los funcionarios                   &
	- Capacitación en uso de \newline herramientas digitales \newline
	- Implementación de programas de bienestar \newline
	- Evaluaciones periódicas                                &
	Talento Humano \newline Supervisores directos                                                        \\ \hline

	Suficiente personal de apoyo                             &
	- Reclutamiento de personal \newline administrativo y técnico \newline
	- Capacitación inicial \newline
	- Definición de roles y turnos \newline
	- Evaluación de desempeño del \newline personal de apoyo &
	Talento Humano \newline Coordinadores de Área                                                        \\ \hline

	Aumento de la infraestructura tecnológica                &
	- Compra de servidores y equipos \newline
	- Modernización del centro de datos \newline
	- Migración a la nube \newline
	- Implementación de ciberseguridad                       &
	Proveedor tecnológico \newline Área de TI \newline Jefes de Área                                     \\ \hline
\end{longtable}

\section{Análisis de alternativas}

\subsection{Alta integridad locativa}
Para la cual se desarrollarán las actividades de diagnóstico de espacios físicos, reestructuración de áreas de trabajo, implementación de mobiliario ergonómico y tecnológico, y adecuaciones eléctricas y de red. En este orden de ideas, para la elaboración de las tareas antes mencionadas, se encargará el área de infraestructura, un arquitecto y proveedores externos respectivamente.

\subsection{Procedimientos automáticos}
Este componente tendrá en cuenta las labores de identificación de procesos, diseño de flujos automatizados, implementación de software, y pruebas y capacitación. Para ello será necesario designar a un ingeniero de sistemas, un grupo de desarrolladores y jefes de área.

\subsection{Alta productividad de los funcionarios}
Se requiere de la capacitación en uso de herramientas digitales, la implementación de programas de bienestar y evaluaciones periódicas, que estarán a cargo del talento humano y los supervisores directos.

\subsection{Suficiente personal de apoyo}
Este componente requerirá reclutamiento de personal administrativo y técnico, capacitación inicial, definición de roles y turnos, y evaluación de desempeño del personal de apoyo. De estas labores se harán cargo el personal de talento humano y los coordinadores de área.

\subsection{Aumento de la infraestructura tecnológica}
Tendrá como labores fundamentales la compra de servidores y equipos, modernización del centro de datos, migración a la nube y la implementación de ciberseguridad, que estarán a cargo de un proveedor tecnológico, área de TI y jefes de área.


\section{Estructura analítica del proyecto (EAP)}

\begin{figure}[H]
	\centering
	\includegraphics[width=0.3\linewidth]{Estructura analítica.png}
	\caption{Estructura analítica del proyecto (EAP).}
	\label{fig:eap}
	\vspace{0.2cm}
	\small Fuente: Elaboración propia.
\end{figure}

\section{Descripción del proyecto}

\subsection{Objetivos}

\subsubsection{Objetivo General}
Implementar un sistema de trámites para licencias de conducción ágil, eficiente y automatizado, que mejore la experiencia del ciudadano.

\subsubsection{Objetivos Específicos}
\begin{itemize}
	\item Automatizar los procedimientos administrativos relacionados con la expedición de licencias
	\item Incrementar la productividad de los funcionarios mediante capacitación y herramientas digitales.
	\item Fortalecer la trazabilidad, seguridad y interoperabilidad del proceso mediante la integración con plataformas externas (RUNT, CRC, pasarelas de pago) y la implementación de mecanismos de contingencia, auditoría y verificación biométrica.
\end{itemize}

\subsection{Alcance}
El presente proyecto tiene por objeto definir, desarrollar e implantar la solución correspondiente a la etapa final
del proceso de expedición de licencias de conducción para personas que solicitan la licencia por primera vez. Esta
intervención se orienta a optimizar la atención presencial en las sedes de la Secretaría de Tránsito, garantizando orden,
trazabilidad y seguridad en la entrega física de la licencia y de los comprobantes asociados. El sistema estará compuesto por
varios módulos funcionales que trabajarán de forma integrada para garantizar una atención ordenada, segura y trazable en las
sedes de la Secretaría de Tránsito. A continuación, se describen los módulos principales que conformarán el sistema:

\subsubsection*{Módulo de gestión de llegada y turnos}
Este módulo atenderá el registro de la llegada de los ciudadanos y la administración de las colas de atención. Sus funciones
principales son:
\begin{itemize}
	\item Registrar la llegada del solicitante y asociarla al expediente correspondiente.
	\item Generar y asignar turnos de atención con estimación de tiempo.
	\item Mostrar la cola de atención para los funcionarios y permitir la llamada ordenada de turnos.
	\item Registrar horarios de inicio y término de la atención para efectos de control y reporte.
\end{itemize}

\subsubsection*{Módulo de verificación documental}
Este módulo permitirá la revisión guiada de los documentos exigidos antes de proceder con la entrega de la licencia. Sus funciones
principales son:
\begin{itemize}
	\item Presentar la lista de documentos mínimos requeridos por tipo de trámite.
	\item Registrar la constatación de cada documento y las observaciones del funcionario.
	\item Bloquear la emisión cuando falte un requisito crítico y generar las opciones de gestión correspondientes.
	\item Permitir la firma del funcionario responsable sobre la verificación realizada.
\end{itemize}

\subsubsection*{Módulo de conciliación de pagos}
Este módulo gestionará la verificación del pago del trámite y la emisión del recibo correspondiente. Sus funciones principales son:
\begin{itemize}
	\item Consultar el estado del pago asociado al expediente.
	\item Registrar la referencia de pago y generar el comprobante oficial.
	\item Gestionar la conciliación manual cuando sea necesario y registrar el resultado de la conciliación.
	\item Bloquear la expedición de la licencia si no se comprueba el pago, salvo autorización expresa.
\end{itemize}

\subsubsection*{Módulo de validación de aptitud médica}
Este módulo se encargará de la verificación de los certificados médicos emitidos por los centros de reconocimiento de conductores. Sus
funciones principales son:
\begin{itemize}
	\item Validar la vigencia y la condición de aptitud del certificado médico.
	\item Registrar el resultado de la validación y notificar las acciones requeridas en caso de no aptitud.
	\item Facilitar la reprogramación de valoraciones cuando proceda.
\end{itemize}

\subsubsection*{Módulo de integración con RUNT y modo de contingencia}
Este módulo permitirá verificar la situación administrativa del solicitante en el registro nacional y gestionar interrupciones del
servicio. Sus funciones principales son:
\begin{itemize}
	\item Consultar el registro nacional para comprobar duplicados, sanciones y datos de identificación.
	\item Bloquear la emisión en caso de inconsistencias y generar procesos de verificación.
	\item Activar un modo de contingencia cuando el servicio externo no esté disponible, con registro de acciones pendientes de sincronización.
\end{itemize}

\subsubsection*{Módulo de generación e impresión de la licencia}
Este módulo será responsable de producir el documento físico final y controlar su entrega. Sus funciones principales son:
\begin{itemize}
	\item Construir el documento oficial con los datos validados del solicitante.
	\item Gestionar la cola de impresión segura y registrar la autorización del funcionario para la impresión.
	\item Registrar el número de serie del documento impreso y conservar la trazabilidad de la emisión.
\end{itemize}

\subsubsection*{Módulo de verificación biométrica}
Este módulo posibilitará la verificación de la identidad cuando así lo exija la normativa o el procedimiento. Sus funciones principales
son:
\begin{itemize}
	\item Capturar las muestras biométricas necesarias en el punto de atención.
	\item Comparar la muestra con los registros disponibles y registrar el resultado del cotejo.
	\item Activar procedimientos de verificación manual en caso de discrepancia o fallos de captura.
\end{itemize}

\subsubsection*{Módulo de gestión de excepciones y reprocesos}
Este módulo permitirá atender y resolver casos que requieran acciones adicionales por inconsistencias o errores. Sus funciones
principales son:
\begin{itemize}
	\item Abrir y gestionar casos con motivo, evidencias y plazos de resolución.
	\item Asignar responsables internos y seguir el estado hasta el cierre del caso.
	\item Registrar todas las actividades relacionadas con el caso para fines de auditoría.
\end{itemize}

\subsubsection*{Módulo de notificaciones y portal ciudadano}
Este módulo facilitará la comunicación con el solicitante y el acceso a comprobantes. Sus funciones principales son:
\begin{itemize}
	\item Enviar notificaciones sobre citas, estado del trámite y entrega de comprobantes.
	\item Generar y poner a disposición comprobantes oficiales en formato descargable.
	\item Mantener un historial de comunicaciones y accesos para control y seguimiento.
\end{itemize}

\subsubsection*{Módulo de integración con proveedores}
Este módulo regulará la recepción de documentos y la interoperabilidad con proveedores externos. Sus funciones principales son:
\begin{itemize}
	\item Recibir y registrar documentos enviados por centros de reconocimiento de conductores y otras entidades autorizadas.
	\item Validar el origen de la información y asociarla al expediente correspondiente.
	\item Gestionar la cuarentena de documentos cuando se detecte alguna inconsistencia en la validación.
\end{itemize}

\subsubsection*{Módulo de gestión de usuarios y permisos}
Este módulo regulará el acceso al sistema y las autorizaciones. Sus funciones principales son:
\begin{itemize}
	\item Administrar cuentas de usuario y asignar roles según las responsabilidades institucionales.
	\item Controlar el acceso a funciones y datos sensibles mediante permisos definidos.
	\item Registrar las acciones de los usuarios para efectos de trazabilidad.
\end{itemize}

\subsubsection*{Módulo de auditoría y reportes}
Este módulo garantizará la trazabilidad y la rendición de cuentas. Sus funciones principales son:
\begin{itemize}
	\item Registrar de forma inalterable las acciones críticas realizadas en el sistema.
	\item Generar reportes operativos y de cumplimiento para la gestión interna.
	\item Proveer información que permita evaluar indicadores de desempeño vinculados al servicio.
\end{itemize}

\subsubsection*{Módulo de retención y conservación documental}
Este módulo asegurará el cumplimiento de las políticas de custodia y disposición de documentos. Sus funciones principales son:
\begin{itemize}
	\item Programar plazos de retención conforme a la normativa vigente.
	\item Controlar la exportación de expedientes a archivo y la eliminación segura cuando proceda.
	\item Mantener registros de todas las acciones relacionadas con la conservación documental.
\end{itemize}

\subsubsection*{Módulo de capacitación y soporte}
Este módulo facilitará la adopción y el uso adecuado del sistema por parte de los funcionarios. Sus funciones principales son:
\begin{itemize}
	\item Ofrecer materiales de formación y guías operativas para los usuarios del sistema.
	\item Proporcionar soporte técnico y procedimientos para la resolución de incidencias.
	\item Programar actividades de actualización y capacitación continua conforme a las necesidades institucionales.
\end{itemize}


\subsection{Población impactada}
El proyecto impactará de manera directa a los ciudadanos que adelantan por primera vez el proceso de obtención de la licencia de conducción, así como a los funcionarios y entidades que intervienen en la etapa final de expedición del documento. En primer lugar, los solicitantes de licencia serán los principales beneficiarios, dado que el sistema propuesto permitirá una atención más ágil, ordenada y transparente. A través de la automatización de los turnos, la verificación documental y la trazabilidad de cada trámite, los usuarios experimentarán una reducción en los tiempos de espera y una mayor claridad sobre el estado de su proceso, fortaleciendo así la confianza en la gestión de la Secretaría de Tránsito.

De igual forma, el proyecto tendrá un impacto positivo sobre los funcionarios de la entidad, especialmente aquellos encargados de la atención al ciudadano, la validación de documentos y la supervisión de los procesos de entrega. La implementación del sistema proporcionará herramientas más eficientes para la revisión, control y registro de cada expediente, reduciendo la carga operativa y mejorando la coordinación entre áreas. Esto favorecerá un entorno de trabajo más ordenado, con procedimientos uniformes y verificables, lo que contribuirá a una gestión institucional más eficiente y trazable.

También se verán beneficiadas las áreas administrativas y financieras de la Secretaría, que contarán con registros más consistentes y conciliaciones automáticas en los procesos de pago y emisión. Este componente permitirá fortalecer el control interno, minimizar los errores humanos y facilitar la rendición de cuentas ante los entes de control. Asimismo, los proveedores externos, como los centros de reconocimiento de conductores y las entidades que transmiten información al Registro Único Nacional de Tránsito (RUNT), tendrán un flujo de comunicación más estable y seguro, evitando reprocesos y pérdidas de información.

\subsection{Elementos normativos}
Mencionar las normas, leyes, reglamentos o estándares aplicables (si existen).

\subsection{Estructura organizacional}
Explicación de la estructura organizativa propuesta para el proyecto.

\subsubsection*{Manual de funciones por cargo}
\begin{itemize}
	\item Cargo 1 – Funciones principales.
	\item Cargo 2 – Funciones principales.
	\item Cargo 3 – Funciones principales.
\end{itemize}

\subsection{Ingeniería del producto}

\subsubsection{Instalaciones locativas}
Descripción de los espacios físicos necesarios.

\subsubsection{Ingeniería de software}

\textbf{Requisitos funcionales}

\begin{enumerate}

	\item \textbf{RF-01: }
	      Cuando un usuario llega a la oficina de la Secretaría para la entrega final de la licencia (ya sea con cita previa o de forma presencial), el sistema debe permitir registrar su llegada, mostrar y administrar su turno, y registrar la hora de inicio y término de la atención. Para esto el sistema toma el identificador del usuario (por ejemplo número de documento) y el expediente electrónico y verifica que los pasos previos estén completos. El operador ve en pantalla el estado del expediente, el sistema asigna un ticket de atención o remarca la cita y mantiene la cola en orden. El resultado es un ticket con sello horario y el expediente con el registro de atención; el sistema debe responder rápido (interfaz visible en tiempos operativos) y dejar trazabilidad del inicio y cierre de la atención. Si el expediente no puede continuar, deberá registrarse el motivo y ofrecer reprogramación o abrir un caso de excepción.

	\item \textbf{RF-02: }
	      Para cada entrega final el sistema debe mostrar una lista de verificación de los documentos requeridos (cédula, comprobante de pago, certificado CRC, resultados de exámenes, constancias RUNT, foto/biometría cuando aplique) y permitir que el funcionario marque cada ítem como “conforme”, “pendiente” o “rechazado”, añadiendo observaciones. El checklist está ligado al expediente electrónico: si falta un documento o está vencido, el sistema bloquea la expedición y genera las opciones correspondientes (reprogramación, recepción parcial con justificación, solicitud de subida de documento). Toda validación queda firmada digitalmente por el funcionario y queda auditada. En casos excepcionales la aceptación sin cumplimiento total solo es posible con justificación y firma del supervisor.

	\item \textbf{RF-03: }
	      Antes de emitir la licencia, el sistema debe consultar el RUNT para verificar que los datos del solicitante, la categoría de la licencia, sanciones o duplicados estén correctamente registrados. Si la consulta indica inconsistencias o duplicados, la emisión queda bloqueada hasta su resolución y el sistema muestra el detalle para que el operador actúe. Si RUNT no está disponible, el sistema activa un modo de contingencia (registro del intento, documento “pendiente de sincronización”) y muestra al funcionario las limitaciones: se registran las acciones realizadas y se exige validación manual supervisada para casos excepcionales.

	\item \textbf{RF-04: }
	      Cuando el expediente esté aprobado, el sistema debe construir el documento final de la licencia en el formato oficial con los datos validados (incluyendo fotografía, número de licencia, firmas digitales y códigos de verificación) y enviar el archivo a una cola de impresión segura. La impresión sólo se libera con la autenticación del funcionario y queda registrada (quién imprimió, cuándo y cuántas copias). El número de serie impreso se sincroniza con el expediente y con RUNT. Cualquier excepción exige doble verificación (operador y supervisor) y se registra para auditoría.

	\item \textbf{RF-05: }
	      La expedición final sólo procede si el pago correspondiente ha sido verificado y conciliado. El sistema debe recibir la referencia de pago, validar el monto y el canal (pasarela en línea o registro presencial), y marcar el expediente como “pagado”. Si la conciliación falla, la emisión se bloquea; excepcionalmente, si existe comprobante físico y autorización, se permite un flujo condicionado que quede registrado y conciliado después, con responsabilidad financiera firmada

	\item \textbf{RF-06}
	      El sistema debe integrar o consultar los resultados emitidos por los CRC para verificar la aptitud médica del solicitante según la categoría de licencia. Al recibir el número de radicado o el archivo, el sistema valida vigencia, parámetros exigidos y marca la condición del certificado (Apto, No apto, Vencido). Si el certificado no es válido o está vencido, la expedición queda bloqueada y se ofrece reprogramación para nueva valoración.

	\item \textbf{RF-07: }
	      Cuando el expediente presente inconsistencias (documentales, errores de sincronización con RUNT, problemas de pago, etc.) el sistema debe permitir crear un caso con workflow: notificar al usuario, permitir la subida de documentos, asignar responsables internos, establecer plazos y registrar el cierre del caso. Todos los reprocesos deben quedar auditables y con SLA internos que permitan medir tiempos de resolución.

	\item \textbf{RF-08: }
	      En cada hito relevante (confirmación o cambio de cita, aceptación de documentos, comprobante de pago, emisión de número de licencia, salida a mensajería) el sistema debe enviar notificaciones al usuario por el canal registrado (correo electrónico, SMS, o ambos) y mantener un log de intentos y entregas. Los comprobantes oficiales incluyen códigos de verificación y se generan en PDF listos para descarga o impresión por parte del usuario.

	\item \textbf{RF-09: }
	      Cuando aplique por normativa o riesgos de suplantación, el sistema debe soportar la verificación biométrica (huella, cotejo facial) en puntos críticos, registrando intentos, resultado del cotejo y la autorización para continuar. Si no hay match biométrico, se debe activar un protocolo de verificación manual por supervisor y registrar la acción con justificación.

	\item \textbf{RF-10: }
	      Siempre que RUNT, la pasarela de pagos u otros servicios externos presenten fallos, el sistema debe permitir operar en modo offline controlado: registrar transacciones y eventos localmente marcados como “pendientes de sincronización”, limitar acciones no permitidas sin conciliación, generar informes de sincronización y forzar la aprobación de un supervisor para acciones excepcionales. Posteriormente, al restablecerse el servicio, el sistema sincroniza y concilia automáticamente los registros.

	\item \textbf{RF-11: }
	      El sistema debe aplicar las reglas de retención documental vigentes: programar plazos de conservación, generar alertas previas a eliminación, soportar la exportación segura a archivo y asegurar eliminación segura cuando proceda. Las políticas de protección de datos personales se aplican a todo el ciclo de vida del expediente.
\end{enumerate}

\textbf{Casos de uso}

\noindent \textbf{UC-01}\\
\textbf{Actores:} Solicitante (usuario), Funcionario, Sistema de turnos.\\
\textbf{Precondición:} El solicitante llega a la oficina para la entrega final (con cita o sin cita).\\
\textbf{Flujo principal:}
\begin{enumerate}
	\item El solicitante se identifica en kiosco/ventanilla mediante documento o QR de trámite.
	\item El sistema busca el expediente asociado y verifica pasos previos (pagos, CRC, resultados de exámenes, RUNT preliminar).
	\item Si todo está correcto, el sistema genera turno y muestra tiempo estimado; imprime ticket si procede.
	\item El sistema registra la hora de llegada en el expediente.
\end{enumerate}
\textbf{Flujos alternativos:}
\begin{itemize}
	\item Si falta documentación, el sistema notifica al funcionario y ofrece opciones: reprogramar, recepción parcial o apertura de caso de excepción.
	\item Si el sistema de turnos está caído, el funcionario puede asignar turno manualmente con registro justificativo (modo contingencia).
\end{itemize}
\textbf{Postcondición:} Turno registrado; expediente actualizado con hora de llegada; usuario notificado del turno.\\
\textbf{RFs relacionadas:} RF-01, RF-10, RF-07.\\
\textbf{Prioridad:} Alta.

\vspace{8pt}

\noindent \textbf{UC-02}\\
\textbf{Actores:} Funcionario, Solicitante, Sistema documental.\\
\textbf{Precondición:} Solicitante en ventanilla con turno activo.\\
\textbf{Flujo principal:}
\begin{enumerate}
	\item El funcionario abre el checklist del expediente en la interfaz.
	\item El sistema presenta la lista de documentos requeridos (cédula, comprobante de pago, CRC, exámenes, foto).
	\item El funcionario marca cada ítem “Conforme / Pendiente / Rechazado”, agrega observaciones y firma digitalmente.
	\item Si todo es “Conforme”, se permite continuar al siguiente paso (RUNT / impresión).
\end{enumerate}
\textbf{Flujos alternativos:}
\begin{itemize}
	\item Si algún documento está vencido o faltante, el sistema bloquea emisión y propone reprogramación o apertura de caso.
\end{itemize}
\textbf{Postcondición:} Checklist guardado y firmado; expediente con estado actualizado.\\
\textbf{RFs relacionadas:} RF-02, RF-14, RF-10.\\
\textbf{Prioridad:} Alta.

\vspace{8pt}

\noindent \textbf{UC-03}\\
\textbf{Actores:} Sistema, Servicio RUNT, Funcionario.\\
\textbf{Precondición:} Checklist aprobado y listo para emitir.\\
\textbf{Flujo principal:}
\begin{enumerate}
	\item El sistema envía consulta en tiempo real a RUNT con datos del solicitante.
	\item RUNT devuelve estado (OK / inconsistencias / duplicados / sanciones).
	\item El sistema presenta el resultado al funcionario; si OK, continúa el flujo; si no, bloquea emisión e informa acciones a tomar.
\end{enumerate}
\textbf{Flujos alternativos:}
\begin{itemize}
	\item Si RUNT no responde, el sistema activa modo contingencia: crea registro “pendiente de sincronización” y habilita verificación manual supervisada.
	\item Si hay duplicado, abrir caso de verificación de identidad y resolución.
\end{itemize}
\textbf{Postcondición:} Resultado de consulta registrado en auditoría; expediente listo o bloqueado según resultado.\\
\textbf{RFs relacionadas:} RF-03, RF-10, RF-09, RF-07.\\
\textbf{Prioridad:} Alta.

\vspace{8pt}

\noindent \textbf{UC-04}\\
\textbf{Actores:} Funcionario, Pasarela de pago, Solicitante.\\
\textbf{Precondición:} Trámite con pago requerido registrado como pendiente.\\
\textbf{Flujo principal:}
\begin{enumerate}
	\item El funcionario ingresa referencia de pago o el sistema consulta la pasarela.
	\item El sistema valida monto y canal, marca el expediente como “Pagado”.
	\item El sistema genera recibo y lo asocia al expediente; notifica al solicitante.
\end{enumerate}
\textbf{Flujos alternativos:}
\begin{itemize}
	\item Si no se encuentra conciliación, el expediente queda bloqueado; el solicitante puede presentar comprobante físico para revisión (emisión condicional con firma de finanzas).
	\item Si la pasarela no responde, registrar intento y operar en modo contingencia.
\end{itemize}
\textbf{Postcondición:} Estado de pago reflejado; recibo disponible para descarga.\\
\textbf{RFs relacionadas:} RF-05, RF-10.\\
\textbf{Prioridad:} Alta.

\vspace{8pt}

\noindent \textbf{UC-05}\\
\textbf{Actores:} Funcionario, CRC, Sistema integrador.\\
\textbf{Precondición:} El solicitante debe presentar certificado médico exigido para primera licencia.\\
\textbf{Flujo principal:}
\begin{enumerate}
	\item El funcionario introduce el número de radicado o el archivo digital del certificado.
	\item El sistema consulta vigencia y parámetros; marca como “Apto / No apto / Vencido”.
	\item Si es “Apto”, continúa el flujo; si no, se informa y se agenda revaluación.
\end{enumerate}
\textbf{Flujos alternativos:}
\begin{itemize}
	\item Si el certificado fue subido por el CRC vía portal, se asocia automáticamente.
	\item Si el sistema CRC no responde, gestionar contingencia y notificar al solicitante.
\end{itemize}
\textbf{Postcondición:} Estado del certificado registrado; bloqueo de emisión si no es apto.\\
\textbf{RFs relacionadas:} RF-06, RF-15, RF-10.\\
\textbf{Prioridad:} Alta.

\vspace{8pt}

\noindent \textbf{UC-06}\\
\textbf{Actores:} Funcionario, Solicitante, Dispositivo biométrico, Sistema biométrico.\\
\textbf{Precondición:} Requisito normativo o riesgo de suplantación identificado; solicitante en ventanilla.\\
\textbf{Flujo principal:}
\begin{enumerate}
	\item El funcionario solicita biometría (huella o foto) al solicitante.
	\item El dispositivo captura la muestra y el sistema compara con el registro asociado en el expediente.
	\item Si hay match, permitir continuar; si no, activar protocolo de verificación manual y notificar supervisor.
	\item Registrar intento y resultado en auditoría.
\end{enumerate}
\textbf{Flujos alternativos:}
\begin{itemize}
	\item Si el dispositivo falla, permitir captura manual y firma de verificación por supervisor.
\end{itemize}
\textbf{Postcondición:} Resultado biométrico registrado; autorización para emisión o inicio de caso de verificación.\\
\textbf{RFs relacionadas:} RF-09, RF-07, RF-10, RF-14.\\
\textbf{Prioridad:} Media.

\vspace{8pt}

\noindent \textbf{UC-07}\\
\textbf{Actores:} Funcionario, Solicitante, Supervisor, Equipo administrativo.\\
\textbf{Precondición:} Detectada inconsistencia documental, fallo en RUNT, pago no conciliado o discrepancia de identidad.\\
\textbf{Flujo principal:}
\begin{enumerate}
	\item El funcionario crea el caso de excepción en el sistema, describiendo motivo y documentos faltantes.
	\item El sistema notifica al solicitante y asigna responsable con plazo.
	\item El solicitante sube documentos o acude según indicaciones; el responsable verifica y cierra el caso.
	\item Caso cerrado: expediente actualizado y registro de acciones.
\end{enumerate}
\textbf{Flujos alternativos:}
\begin{itemize}
	\item Si el solicitante no cumple dentro del plazo, el sistema genera alertas y eventualmente cierra el caso por inactividad.
	\item Para reprocesos por falla del sistema, el supervisor puede priorizar la atención al reencolar el expediente.
\end{itemize}
\textbf{Postcondición:} Caso documentado y auditable; resolución aplicada al expediente.\\
\textbf{RFs relacionadas:} RF-07, RF-01, RF-03, RF-05.\\
\textbf{Prioridad:} Alta.

\vspace{8pt}

\noindent \textbf{UC-08}\\
\textbf{Actores:} Sistema de emisión, Funcionario, Impresora certificada.\\
\textbf{Precondición:} Expediente aprobado, pago conciliado, CRC apto, RUNT OK.\\
\textbf{Flujo principal:}
\begin{enumerate}
	\item El sistema construye el documento final con fotografía, datos y códigos de verificación.
	\item El funcionario autentica la liberación de impresión.
	\item El sistema envía a la cola de impresión segura; la impresión se registra (operador, hora, nº de serie).
	\item El sistema sincroniza el número de serie con el expediente y con RUNT.
\end{enumerate}
\textbf{Flujos alternativos:}
\begin{itemize}
	\item Si la impresora falla, marcar reintento y guardar archivo maestro; posibilidad de impresión en otra estación con autorización.
	\item Si existe excepción, requerir firma adicional antes de liberar impresión.
\end{itemize}
\textbf{Postcondición:} Licencia física impresa y asociada al expediente; registro de emisión.\\
\textbf{RFs relacionadas:} RF-04, RF-10, RF-14.\\
\textbf{Prioridad:} Alta.

\vspace{8pt}

\noindent \textbf{UC-09}\\
\textbf{Actores:} Sistema de notificaciones, Solicitante, Funcionario.\\
\textbf{Precondición:} Cualquier hito importante (cita, recepción documento, pago, emisión, envío).\\
\textbf{Flujo principal:}
\begin{enumerate}
	\item El sistema selecciona plantilla y canal (correo o SMS) según preferencia registrada.
	\item Envía la notificación con datos y/o adjunta comprobantes en PDF.
	\item Registra la entrega (éxito/fracaso) en el log de comunicaciones.
\end{enumerate}
\textbf{Flujos alternativos:}
\begin{itemize}
	\item Si el envío falla, el sistema reintenta según política y registra intentos; informa al funcionario si hay fallas repetidas.
\end{itemize}
\textbf{Postcondición:} Notificación enviada y registro de intento en expediente.\\
\textbf{RFs relacionadas:} RF-08.\\
\textbf{Prioridad:} Media.

\vspace{8pt}

\noindent \textbf{UC-10}\\
\textbf{Actores:} Funcionario, Supervisor, Sistemas externos (RUNT, pasarela), Sistema local.\\
\textbf{Precondición:} Caída o indisponibilidad de RUNT, pasarela de pagos u otro servicio crítico.\\
\textbf{Flujo principal:}
\begin{enumerate}
	\item El sistema detecta el fallo en el servicio externo y activa modo contingencia.
	\item El funcionario puede continuar ciertos pasos (registro local de eventos, captura de comprobantes) pero con limitaciones (no se permite emisión final si no cumple reglas).
	\item Todas las acciones quedan marcadas como “pendientes de sincronización”.
	\item Al restablecerse el servicio, el sistema sincroniza y concilia automáticamente los registros pendientes.
\end{enumerate}
\textbf{Flujos alternativos:}
\begin{itemize}
	\item Para casos excepcionales que requieren emisión inmediata y el servicio sigue caído, el supervisor puede autorizar emisión manual con justificativo y firma electrónica.
\end{itemize}
\textbf{Postcondición:} Registros en modo offline, listas para sincronizar; trazabilidad completa.\\
\textbf{RFs relacionadas:} RF-10, RF-07, RF-04, RF-05, RF-03.\\
\textbf{Prioridad:} Alta.

\vspace{8pt}

\noindent \textbf{UC-11}\\
\textbf{Actores:} Sistema de auditoría, Auditor/Administrador, Funcionario.\\
\textbf{Precondición:} Cualquier acción relevante sobre expediente (consultas, impresiones, cambios de estado).\\
\textbf{Flujo principal:}
\begin{enumerate}
	\item El sistema registra el evento apend-only con usuario, timestamp, motivo y metadatos.
	\item El administrador/auditor puede consultar y exportar reportes filtrados por periodo, sede o tipo de evento.
	\item El sistema aplica controles de acceso a los logs y preserva integridad.
\end{enumerate}
\textbf{Flujos alternativos:}
\begin{itemize}
	\item Si hay intento de eliminación o modificación ilícita, el sistema genera alerta y bloquea la operación; registro forense disponible.
\end{itemize}
\textbf{Postcondición:} Eventos auditados y reportes disponibles.\\
\textbf{RFs relacionadas:} RF-10, RF-11.\\
\textbf{Prioridad:} Alta.

\vspace{8pt}

\noindent \textbf{UC-12}\\
\textbf{Actores:} CRC/Escuela de conducción/Entidad financiera, Sistema, Funcionario.\\
\textbf{Precondición:} Proveedor autorizado sube certificado/comprobante.\\
\textbf{Flujo principal:}
\begin{enumerate}
	\item El proveedor envía documento y metadatos mediante portal seguro.
	\item El sistema valida la firma y asocia el documento al expediente correspondiente.
	\item El funcionario recibe notificación y el documento aparece en el checklist.
\end{enumerate}
\textbf{Flujos alternativos:}
\begin{itemize}
	\item Si la firma no es válida o el proveedor no está autorizado, el documento queda en cuarentena y se notifica a supervisión.
\end{itemize}
\textbf{Postcondición:} Documento disponible en expediente y listado en checklist.\\
\textbf{RFs relacionadas:} RF-15, RF-02.\\
\textbf{Prioridad:} Media-Alta.

\vspace{8pt}

\noindent \textbf{UC-13}\\
\textbf{Actores:} Administrador, Sistema de archivo, Auditoría.\\
\textbf{Precondición:} Plazo de retención próximo o petición de archivo/eliminación.\\
\textbf{Flujo principal:}
\begin{enumerate}
	\item El sistema identifica expedientes próximos a vencimiento de retención.
	\item Envía alertas a administración para revisar y autorizar exportación o eliminación.
	\item Si procede, exporta a archivo o elimina según política, dejando registro en auditoría.
\end{enumerate}
\textbf{Flujos alternativos:}
\begin{itemize}
	\item Si el expediente está sujeto a investigación o requerimiento legal, se suspende eliminación y se marca retención extendida.
\end{itemize}
\textbf{Postcondición:} Registro de conservación/eliminación y cumplimiento documental.\\
\textbf{RFs relacionadas:} RF-11.\\
\textbf{Prioridad:} Media.

\vspace{8pt}

\noindent \textbf{UC-14}\\
\textbf{Actores:} Solicitante, Sistema de expedientes.\\
\textbf{Precondición:} Hito cumplido (pago, emisión, notificación).\\
\textbf{Flujo principal:}
\begin{enumerate}
	\item El solicitante accede al portal con credenciales o con código seguro.
	\item Selecciona comprobante y descarga PDF con sello y código de verificación.
	\item El sistema registra la descarga en auditoría.
\end{enumerate}
\textbf{Flujos alternativos:}
\begin{itemize}
	\item Si el usuario no figura o hay múltiples expedientes, se solicita verificación.
\end{itemize}
\textbf{Postcondición:} Comprobante descargado y registro en log.\\
\textbf{RFs relacionadas:} RF-08, RF-16.\\
\textbf{Prioridad:} Media.

\newpage
\textbf{Diagrama de secuencia}

\begin{figure}[H]
	\centering
	\includegraphics[width=1\linewidth]{DiagramSecuence.png}
	\caption{Diagrama de secuencia}
	\vspace{0.5cm}
	\small Fuente: Elaboración propia.
	\label{fig:sequence}
\end{figure}

\newpage

\textbf{Diagramas de actividades}

\begin{figure}[H]
	\centering
	\includegraphics[width=0.5\linewidth]{ActivityDiagram1.png}
	\caption{Diagrama de actividades 1}
	\vspace{0.5cm}
	\small Fuente: Elaboración propia.
	\label{fig:activity1}
\end{figure}

\begin{figure}[H]
	\centering
	\includegraphics[width=1\linewidth]{ActivityDiagram2.png}
	\caption{Diagrama de actividades 2}
	\vspace{0.5cm}
	\small Fuente: Elaboración propia.
	\label{fig:activity2}
\end{figure}

\begin{figure}[H]
	\centering
	\includegraphics[width=0.5\linewidth]{ActivityDiagram3.png}
	\caption{Diagrama de actividades 3}
	\vspace{0.5cm}
	\small Fuente: Elaboración propia.
	\label{fig:activity3}
\end{figure}

\newpage

\textbf{Diagrama de clases}

\begin{figure}[H]
	\centering
	\includegraphics[width=0.9\linewidth]{ClassDiagram.png}
	\caption{Diagrama de clases}
	\vspace{0.5cm}
	\small Fuente: Elaboración propia.
	\label{fig:classdiagram}
\end{figure}

\textbf{Diagrama de componentes}
\begin{figure}[H]
	\centering
	\includegraphics[width=0.9\linewidth]{DiagramadeComponentes.png}
	\caption{Diagrama de componentes}
	\vspace{0.5cm}
	\small Fuente: Elaboración propia.
	\label{fig:Componentdiagram}
\end{figure}

\subsection{Costos detallados y proyectados}
Los costos se presentarán en un archivo de Excel con el desglose correspondiente.


\end{document}
